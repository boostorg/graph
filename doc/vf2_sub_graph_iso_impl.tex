\documentclass[12pt]{article}
\usepackage{amsmath}
\usepackage{algorithmicx}
\usepackage{algpseudocode}
\usepackage{listings}

% abbreviations and acronyms
\def\etal{{et al.\ }}
\def\eg{{\it e.g.\ }}
\def\etc{{\it etc.\ }}
\def\ie{{\it i.e.\ }}
\def\cf{{\it cf.\ }}

\title{An Implementation of the {\sc VF2} (Sub)Graph Isomorphism Algorithm
Using The Boost Graph Library}
\author{Flavio De Lorenzi\thanks{E-mail: fdlorenzi@gmail.com}}

\date{\today}

\begin{document}
\maketitle
%\VerbatimFootnotes

\begin{abstract}
This article describes an implementation of the {\sc VF2} algorithm, introduced
by Cordella \etal for solving the graph isomorphism and graph-subgraph
isomorphism problems, using the Boost Graph Library.  This implementation
includes algorithmic improvements to account for self-loops and works for
directed and undirected graphs.
\end{abstract}

\baselineskip=\normalbaselineskip

%\newpage

\section{Introduction}
This section briefly outlines the {\sc VF2} algorithm\footnote{The original
code by Pasquale Foggia and collaborators can be obtained from:
\texttt{http://www.cs.sunysb.edu/{\textasciitilde}algorith/implement/vflib/implement.shtml}},
following closely \cite{cordella+2001, cordella+2004}.

An isomorphism between two graphs $G_1=(V_1, E_1)$ and $G_2=(V_2, E_2)$ is a
bijective mapping $M$ of the vertices of one graph to vertices of the other
graph that preserves the edge structure of the graphs. $M$ is said to be a
graph-subgraph isomorphism iff $M$ is an isomorphism between $G_1$ and a
subgraph of $G_2$.

A matching process between the two graphs $G_1$ and $G_2$ determines the
isomorphism mapping $M$ which associates vertices of $G_1$ with vertices of
$G_2$ and vice versa.  The matching process can be described by means of a
state space representation combined with a depth-first strategy. The details
can be found in \cite{cordella+2001, cordella+2004} and references therein.

Cordella \etal give the following high-level description of the matching
algorithm:

\begin{algorithmic}
\Procedure{Match}{$s$}
\If {$M(s)$ covers all the nodes of $G_2$}
    \State \Return $M(s)$
\Else
    \State Compute the set $P(s)$ of the pairs of vertices for inclusion in $M(s)$
    \ForAll {$(v,w) \in P(s)$}
        \If {$F(s,v,w)$}
            \State Compute the state $s'$ obtained by adding $(v,w)$ to $M(s)$
            \State \Call {Match}{$s'$}
         \EndIf
    \EndFor
    \State Restore data structures
\EndIf
\EndProcedure
\end{algorithmic}

$M(s)$ is a partial mapping associated with a state $s$, $P(s)$ is the set
of all possible pairs of vertices $(v,w)$ to be added to the current state $s$
and $F(s,v,w)$ is a boolean function (called {\em feasibility function}) used
to prune the search tree. If its value is {\em true} the state $s'$ obtained
by adding $(v,w)$ to $s$ is guaranteed to be a partial isomorphism if $s$ is. 

% \subsubsection*{State Space Representation}

To construct $P(s)$ and $F(s,v,w)$ Cordella \etal define the 
{\em out-terminal set} as the set of vertices of $G_1$ that are not in $M(s)$
but are successors of vertices in $M(s)$ (connected by out edges), and the
{\em in-terminal set} as the set of vertices that are not in $M(s)$ but are
predecessors of vertices in $M(s)$. Analogue sets are defined for $G_2$.

% \subsubsection*{Data Structures}

To compute $P(s)$ and $F(s,v,w)$ efficiently, Cordella \etal employ the
following data structures:

\begin{itemize}

\item Vectors \verb+core_1+ and \verb+core_2+ whose dimensions correspond to
the number of vertices in $G_1$ and $G_2$, respectively. These vectors store the
present mapping.

\item Vectors \verb+in_1+, \verb+out_1+, \verb+in_2+ and \verb+out_2+ used to
describe the membership to the terminal sets. \verb+in_1+ is non-zero for a
particular vertex if the vertex is either in the partial mapping $M(s)$ or
belongs to the in-terminal state of $G_1$. The actual value is given by the
level of the depth-first search tree at which the vertex was included in the
corresponding set.

\end{itemize}


\section{Implementation}

The computations of the terminal sets or the addition (deletion) of a pair of
vertices to (from) a state are analogous for the two graphs $G_1$ and $G_2$.
For example, to add the vertex pair $(v, w)$ with $v \in V_1$ and $w \in V_2$
to vector \verb+core_1+ is the same as adding $(w, v)$ to \verb+core_2+.  This
observation suggests the following improvement to the original {\sc VF2}
implementation.  Instead of implementing a state for $G_1$ and $G_2$ with
associated vectors \verb+core_1+, \verb+core_2+, \verb+in_1+, \verb+out_1+,
\verb+in_2+ and \verb+out_2+ directly, we implement a ``helper state'' class
\verb+base_state+ associated with a single graph. Class \verb+base_state+ then
contains \verb+core_+, \verb+in_+ and \verb+out_+, and member functions such as
%\eg \verb+push(const vertex_this_type& v_this, const vertex_other_type& v_other)+ 
\eg \texttt{push(const vertex \_this\_type\& v\_this, const vertex\_other\_type\& v\_other)} 
to add a vertex pair. The actual state associated with both graphs (implemented in
class \verb+state+) can thus be constructed using two ``helper states'', one
for each graph. For instance, the member function \verb+push+ to add a pair of
vertices to the actual state is obtained as illustrated in the code
fragment below:

\lstset{breaklines=true, breakatwhitespace=true}
\lstset{columns=fullflexible}
\lstset{language=C++}
\begin{lstlisting}
<template<typename Graph1,
          typename Graph2,
          typename IndexMap1,
          typename IndexMap2, .... >
class state
{

    ...

    base_state<Graph1, Graph2, IndexMap1, IndexMap2> state1_;
    base_state<Graph2, Graph1, IndexMap2, IndexMap1> state2_;

public:
    // Add vertex pair to the state
    void push(const vertex1_type& v, const vertex2_type& w)
    {
        state1_.push(v, w);
        state2_.push(w, v);
    }

    ...

};
\end{lstlisting}

These classes (\verb+base_state+ and \verb+state+) and the non-recursive
matching procedure \verb+match+ are all members of namespace \verb+boost::detail+.

The functions of the public interface are all defined in namespace \verb+boost+ and their
documentation will follow in the sections bellow.

\subsection{Functions for Graph Sub-Graph Isomorphism Testing}

\begin{lstlisting}
// Non-named parameter version
template <typename GraphSmall,
          typename GraphLarge,
          typename IndexMapSmall,
          typename IndexMapLarge,
          typename VertexOrderSmall,
          typename EdgeCompatibilityPredicate,
          typename VertexCompatibilityPredicate,
          typename SubGraphIsoMapCallBack>
bool vf2_sub_graph_iso(const GraphSmall& graph_small,
                       const GraphLarge& graph_large,
                       SubGraphIsoMapCallBack user_callback,
                       IndexMapSmall index_map_small,
                       IndexMapLarge index_map_large, 
                       const VertexOrderSmall& vertex_order_small,
                       EdgeCompatibilityPredicate edge_comp,
                       VertexCompatibilityPredicate vertex_comp)
\end{lstlisting}


\begin{lstlisting}
// Named parameter interface of vf2_sub_graph_iso
template <typename GraphSmall,
          typename GraphLarge,
          typename VertexOrderSmall,
          typename SubGraphIsoMapCallBack,
          typename Param,
          typename Tag,
          typename Rest>
bool vf2_sub_graph_iso(const GraphSmall& graph_small,
                       const GraphLarge& graph_large,
                       SubGraphIsoMapCallBack user_callback,
                       const VertexOrderSmall& vertex_order_small,
                       const bgl_named_params<Param, Tag, Rest>& params)
\end{lstlisting}


\begin{lstlisting}
// All default interface for vf2_sub_graph_iso
template <typename GraphSmall,
          typename GraphLarge,
          typename SubGraphIsoMapCallBack>
bool vf2_sub_graph_iso(const GraphSmall& graph_small,
                       const GraphLarge& graph_large, 
                       SubGraphIsoMapCallBack user_callback)
\end{lstlisting}

This algorithm finds all graph-subgraph isomorphism mappings between graphs
\verb+graph_small+ and \verb+graph_large+ and outputs them to \verb+user_callback+.
It continues until \verb+user_callback+ returns true or the search space has
been fully explored.\\
\verb+EdgeCompatibilityPredicate+ and \verb+VertexCompatibilityPredicate+
predicates are used to test whether edges and vertices are compatible.
By default \verb+always_compatible+ is used, which returns true for any pair of
vertices or edges.

\subsubsection*{Parameters}

\begin{itemize}

\item[IN:] \verb+const GraphSmall& graph_small+ The (first) smaller graph (fewer vertices)
of the pair to be tested for isomorphism. The type \verb+GraphSmall+ must be a
model of {\em Vertex List Graph}, {\em Bidirectional Graph}, {\em Edge List
Graph} and {\em Adjacency Matrix}.


\item[IN:] \verb+const GraphLarge& graph_large+ The (second) larger graph to be tested.
Type \verb+GraphLarge+ must be a model of 
{\em Vertex List Graph}, {\em Bidirectional Graph}, {\em Edge List Graph} and
{\em Adjacency Matrix}.

\item[OUT:] \verb+SubGraphIsoMapCallBack user_callback+ A function object to be
called when a graph-subgraph isomorphism has been discovered. The
\verb+operator()+ must have following form:
\begin{lstlisting}
template <typename CorrespondenceMap1To2,
          typename CorrespondenceMap2To1>
bool operator()(CorrespondenceMap1To2 f, CorrespondenceMap2To1 g) const
\end{lstlisting}

Both the \verb+CorrespondenceMap1To2+ and \verb+CorresondenceMap2To1+ types
are models of {\em Readable Property Map} and map equivalent vertices across
the two graphs given to \verb+vf2_sub_graph_iso+ (or \verb+vf2_graph_iso+). An
example is given below. 

Returning false from the callback will abort the search immediately. Otherwise,
the entire search space will be explored.


\item[IN:] \verb+const VertexOrderSmall& vertex_order_small+ The ordered
vertices of the smaller graph \verb+graph_small+. During the matching process the
vertices are examined in the order given by \verb+vertex_order_small+. Type
\verb+VertexOrderSmall+ must be a model of \verb+ContainerConcept+ with value
type \verb+graph_traits<GraphSmall>::vertex_descriptor+.
\\
{\em Default:} The vertices are ordered by multiplicity of in/out degree.

\end{itemize}

\subsubsection*{Named Parameters}
               
\begin{itemize}

\item[IN:] \verb+vertex_index1(IndexMapSmall index_map_small)+
This maps each vertex to an integer in the range \verb+[0, num_vertices(graph_small))+.
\\Type \verb+IndexMapSmall+ must be a model of {\em Readable Property Map}. 
\\
{\em Default:} \verb+get(vertex_index, graph_small)+

\item[IN:] \verb+vertex_index2(IndexMapLarge index_map_large)+ 
This maps each vertex to an integer in the range \verb+[0, num_vertices(graph_large))+.
\\Type \verb+IndexMapLarge+ must be a model of {\em Readable Property Map}. 
\\
{\em Default:} \verb+get(vertex_index, graph_large)+

\item[IN:] \verb+edges_equivalent(EdgeCompatibilityPredicate edge_comp)+
This function object is used to determine if edges between the two graphs 
\verb+graph_small+ and \verb+graph_large+ are compatible.\\ 
Type \verb+EdgeCompatiblePredicate+ must be a model of {\em Binary
Predicate} and have argument types of  
\verb+graph_traits<GraphSmall>::edge_descriptor+ and
\verb+graph_traits<GraphLarge>::edge_descriptor+. A return value of true
indicates that the edges are compatible.\\
{\em Default:} \verb+always_compatible+

\item[IN:] \verb+vertices_equivalent(VertexCompatibilityPredicate vertex_comp)+
This function object is used to determine if vertices between the two graphs 
\verb+graph_small+ and \verb+graph_large+ are compatible.\\ 
Type \verb+VertexCompatiblePredicate+ must be a model of {\em Binary
Predicate} and have argument types of\\
\verb+graph_traits<GraphSmall>::vertex_descriptor+ and\\
\verb+graph_traits<GraphLarge>::vertex_descriptor+. A return value of true
indicates that the vertices are compatible. 
\\
{\em Default:} \verb+always_compatible+

\end{itemize}


\subsection{Functions for Isomorphism Testing}

Non-named parameter, named-parameter and all default parameter versions of 
function
\begin{lstlisting}
vf2_graph_iso(...)
\end{lstlisting}

for isomorphism testing take the same parameters as the corresponding
functions \verb+vf2_sub_graph_iso+. The algorithm finds all isomorphism
mappings between graphs \verb+graph1+ and \verb+graph2+ and outputs them to
\verb+user_callback+. It continues until \verb+user_callback+ returns true
or the search space has been fully explored. As before,
\verb+EdgeCompatibilityPredicate+ and\\
\verb+VertexCompatibilityPredicate+
predicates are used to test whether edges and vertices are compatible with
\verb+always_compatible+ as default.

\subsection{Utility Functions and Structs}

\begin{lstlisting}
template <typename PropertyMap1,
          typename PropertyMap2>
property_map_compatible<PropertyMap1, PropertyMap2>
make_property_map_compatible(const PropertyMap1 property_map1,
                             const PropertyMap2 property_map2) 
\end{lstlisting}
Returns a binary predicate function object \\
(\verb+property_map_compatible<PropertyMap1, PropertyMap2>+) that compares
vertices or edges between graphs using property maps. 


\begin{lstlisting}
struct always_compatible
\end{lstlisting}
A binary function object that returns true for any pair of items. 


\begin{lstlisting}
template <typename Graph1,
          typename Graph2>
struct vf2_print_callback
\end{lstlisting}
Callback function object that prints out the correspondences between vertices
of \verb+Graph1+ and \verb+Graph2+. The constructor takes the two graphs $G_1$
and $G_2$ and an optional \verb+bool+ parameter as arguments. If the latter is
set to \verb+true+, the callback function will verify the mapping before outputting
it to standard output.

\begin{lstlisting}
// Verifies a graph (sub)graph isomorphism map 
template<typename Graph1,
         typename Graph2,
         typename CorresponenceMap1To2,
         typename EdgeCompatibilityPredicate,
         typename VertexCompatibilityPredicate>
inline bool verify_vf2_sub_graph_iso(const Graph1& graph1,
                                     const Graph2& graph2, 
                                     const CorresponenceMap1To2 f,
                                     EdgeCompatibilityPredicate edge_comp, 
                                     VertexCompatibilityPredicate vertex_comp)
\end{lstlisting}


\begin{lstlisting}
// Variant of verify_sub_graph_iso with all default parameters
template<typename Graph1,
         typename Graph2,
         typename CorresponenceMap1To2>
inline bool verify_vf2_sub_graph_iso(const Graph1& graph1,
                                     const Graph2& graph2, 
                                     const CorresponenceMap1To2 f)
\end{lstlisting}

This function can be used to verify a (sub)graph isomorphism mapping {\em f}.
The parameters are akin to function \verb+vf2_sub_graph_iso+
(\verb+vf2_graph_iso+).


\subsection{Complexity}

Spatial and time complexity are given in \cite{cordella+2004}. The spatial
complexity of {\sc VF2} is of order $O(V)$, where $V$ is the (maximum) number
of vertices of the two graphs. Time complexity is $O(V^2)$ in the best case and
$O(V!V)$ in the worst case.

\subsection{A Graph Sub-Graph Isomorphism Example}

In the example below, a small graph \verb+graph1+ and a larger graph
\verb+graph2+ are defined. \\ \verb+vf2_sub_graph_iso+ computes all the
mappings between the two graphs and outputs them via \verb+callback+.


\begin{lstlisting}
typedef adjacency_list<vecS, vecS, bidirectionalS> graph_type;

// Build graph1
int num_vertices1 = 8; graph_type graph1(num_vertices1);
add_edge(0, 6, graph1); add_edge(0, 7, graph1);
add_edge(1, 5, graph1); add_edge(1, 7, graph1);
add_edge(2, 4, graph1); add_edge(2, 5, graph1); add_edge(2, 6, graph1);
add_edge(3, 4, graph1);

// Build graph2
int num_vertices2 = 9; graph_type graph2(num_vertices2);
add_edge(0, 6, graph2); add_edge(0, 8, graph2);
add_edge(1, 5, graph2); add_edge(1, 7, graph2);
add_edge(2, 4, graph2); add_edge(2, 7, graph2); add_edge(2, 8, graph2);
add_edge(3, 4, graph2); add_edge(3, 5, graph2); add_edge(3, 6, graph2);

// true instructs callback to verify a map using
// verify_vf2_sub_graph_iso
vf2_print_callback<graph_type, graph_type> callback(graph1, graph2, true);

bool ret = vf2_sub_graph_iso(graph1, graph2, callback);
\end{lstlisting}

\appendix
\section{Testing}

Also included are \verb+vf2_sub_graph_iso_gviz_example.cpp+ and a Scilab \newline
(\verb+http://www.scilab.org/+) script \verb+vf2_random_graphs.sce+ for testing the
implementation. The script generates pairs of simple graphs of (possibly) different
size, such that there exists at least one (sub)graph isomorphism mapping
between the two graphs. The graphs are written to files \verb+graph_small.dot+
and \verb+graph_large.dot+ using the Graphviz {\em DOT} language
(\verb+http://www.graphviz.org+).  The following parameters can be used to
control the output:

\begin{itemize}

\item \verb+nbig+ Dimension of the large adjacency matrix
\item \verb+nsmall+ Dimension of the small adjacency matrix
\item \verb+density+ Density of the non-zero entries (of an initial square
matrix with dimension \verb+nbig+)
\item \verb+directed+ If set to one, a pair of directed graphs is generated,
otherwise undirected graphs are produced.
\item \verb+loops+ If set to one, self-loops are allowed, otherwise self-loops
are excluded.
\end{itemize}

The generated dot-files specifying the graphs can be given as command line
arguments to the executable test program, which uses boost's GraphViz input
parser to build the graphs. The graphs are then tested for (sub)graph
isomorphism. The isomorphism mappings are verified and written to standard
output.

To build the test executable, you will need to build and link against the
"boost\_graph" and "boost\_regex" libraries, \cf also \verb+read_graphviz+.

%%%%%%%%%%%%%%%
% Bibliography
%%%%%%%%%%%%%%
\begin{thebibliography}{10}

\bibitem{cordella+2001} L. P. Cordella, P. Foggia, C. Sansone, and M. Vento,
    ``An improved algorithm for matching large graphs,'' \emph{In: 3rd IAPR-TC15
    Workshop on Graph-based Representations in Pattern Recognition, Cuen}, 
    pp. 149--159, 2001.

\bibitem{cordella+2004} L. P. Cordella, P. Foggia, C. Sansone, and M. Vento,
    ``A (Sub)Graph Isomorphism Algorithm for Matching Large Graphs,''
    \emph{IEEE Trans. Pattern Anal. Mach. Intell.}, vol. 26, no. 10,
    pp. 1367--1372, 2004

\end{thebibliography}


\end{document}
